\documentclass[a4paper,12pt]{article}

\usepackage{hyperref}

\hypersetup{
  pdftitle={Flying Fist},
  pdfauthor={Kimmo Puputti, Morocz Adam},
  pdfsubject={},
  pdfcreator={},
  pdfproducer={pdflatex},
  pdfkeywords={},
  colorlinks=true,
  linkcolor=red,
  citecolor=green,
  filecolor=magenta,
  urlcolor=blue
}

\title{Web Technology 2II25:\\Flying Fist - final report}
\author{Kimmo Puputti, kimmo.puputti@tkk.fi, 0735552\\M\'orocz \'Ad\'am, morocz@gmail.com, }

\begin{document}

\maketitle

\section{Introduction}

Everything that people do is somehow related to geographical
information. This is why a lot of the resources in WWW contain
geographical information. Place names, however, can be very ambiguous
and it can be hard to distinguish places from text
content. Geographical data is very linked within itself and
geographically annotated data can be easily linked together.

Our application lets you browse geographical data that is linked
within itself and also to other third party sites like Wikipedia
etc. We use the data from Geonames (\url{http://www.geonames.org/})
for places inside The Netherlands in the backend of our application.

You can search for places by their name and browse through the pages
of each place. Every place is linked to several other places. You can
also see the places on a interactive map.

\section{Tool and Technologies}

\subsection{Backend}



\subsection{Frontend}



\end{document}
