\documentclass[a4paper,12pt]{article}

\usepackage{hyperref}

\hypersetup{
  pdftitle={Flying Fist},
  pdfauthor={Kimmo Puputti, Morocz Adam},
  pdfsubject={},
  pdfcreator={},
  pdfproducer={pdflatex},
  pdfkeywords={},
  colorlinks=true,
  linkcolor=red,
  citecolor=green,
  filecolor=magenta,
  urlcolor=blue
}

\title{Web Technology 2II25:\\Flying Fist - final report}
\author{Kimmo Puputti, kimmo.puputti@tkk.fi, 0735552\\M\'orocz \'Ad\'am, morocz@gmail.com, }

\begin{document}

\maketitle

\section{Introduction}

Everything that people do is somehow related to geographical
information. This is why a lot of the resources in WWW contain
geographical information. Place names, however, can be very ambiguous
and it can be hard to distinguish places from text
content. Geographical data is very linked within itself and
geographically annotated data can be easily linked together.

Our application lets you browse geographical data that is linked
within itself and also to other third party sites like Wikipedia
etc. We use the data from Geonames (\url{http://www.geonames.org/})
for places inside The Netherlands in the backend of our application.

You can search for places by their name and browse through the pages
of each place. Every place is linked to several other places. You can
also see the places on a interactive map.

\section{Data and the ontology}



\section{Tool and Technologies}

\subsection{Backend}

The web application is built in Python using CherryPy HTTP framework
and Mako templates. The data is converted into RDF and queried using
RDFLib. The triple store is read into memory from the converted N3
files every time the application is started. We also use PyLucene
(\url{http://lucene.apache.org/pylucene/}) to index and search the
place names.

\subsection{Frontend}

The HTML is built using Mako templates. We use jQuery JavaScript
library (\url{http://jquery.com/}) for the dynamic frontend
functionality and AJAX calls and Google Maps JavaScript API V3
(\url{http://code.google.com/apis/maps/documentation/javascript/}) for
the interactive map.

The layout of the application is done using CSS.

\section{Using the application}

\subsection{Setup}

\noindent \textit{Prerequisites}: CherryPy, Mako, PyLucene, RDFLib,
and RDF Extras (from RDFLib) have to be installed and found on the
Python search path.\\

\noindent To run the application, you first have to clone the Git
repository with the following command:

\begin{verbatim}
git clone git://github.com/kpuputti/webtech.git
\end{verbatim}

\noindent which creates a directory `webtech' that has the `project'
subdirectory with all the source code.

Second, some directories have to be made into the project directory
using the command:

\begin{verbatim}
mkdir data index log output
\end{verbatim}

\noindent Third, the following files must be downloaded from the
Geonames download server
(\url{http://download.geonames.org/export/dump/}) and extracted into
the `data' directory:

\begin{verbatim}
NL.zip
admin1Codes.txt
admin2Codes.txt
countryInfo.txt
featureCodes_en.txt
\end{verbatim}

\noindent Then using the control script \texttt{fist.py} in the
project root folder, the RDF triple store and the Lucene index are
created with the commands:

\begin{verbatim}
python fist.py create_storage
python fist.py create_index
\end{verbatim}

\noindent The RDF graphs are stored in the `output' directory and the
Lucene index in the `index' directory. The log file for the
application is created in the `log' directory and can be followed to
get information of the progress or about errors that might have
happened.

If the backend storages are created successfully, the application can
be started with the command:

\begin{verbatim}
python fist.py run
\end{verbatim}

\subsection{User Interface}



\end{document}
